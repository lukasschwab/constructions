\documentclass{standalone}
\usepackage{tikz}
\usetikzlibrary{arrows, intersections}

% This would be Figure 9, which is not described in Roesling's introduction. I
% assume this is the foiled arch.
% 
% **Figs. 5, 6, 7, 8 - The Ogee, or Contrasted Arches, and the Foiled Arch.** 
% These arches are described from three to five centers. See lines of 
% construction with the centerpoints at their extremities.

\begin{document}
\begin{tikzpicture}
    % Base.
    \draw (0, 0) to (4, 0) ;
    \draw (0, 0) -- (0, 4) ;
    \draw (4, 0) -- (4, 4) ;

    % First horizontal guide with compass points.
    \draw [gray, dotted] (0, 4) to (1, 4) node[circle, draw, solid] {} ;
    \draw [gray, dotted] (1, 4) to (3, 4) node[circle, draw, solid] {} ;
    % Guide grid.
    \draw [gray, dotted] (1, 6) to (3, 6) ;
    \draw [gray, dotted] (1, 5) to (3, 5) ;
    \draw [gray, dotted] (2, 4) to (2, 6) ;
    % Upper compass points.
    \draw [gray, dotted] (1, 4) to (1, {5+sqrt(3)}) node[circle, draw, solid] {};
    \draw [gray, dotted] (3, 4) to (3, {5+sqrt(3)}) node[circle, draw, solid] {} ;
    \draw [gray, dotted] (1, {5+sqrt(3)}) to (3, {5+sqrt(3)}) ;
    % Central compass point.
    \draw [gray, dotted] (1, {5+sqrt(3)}) to (2, 5) node[circle, draw, solid] {};
    \draw [gray, dotted] (3, {5+sqrt(3)}) to (2, 5);

    % Arch: lower.
    \draw (0, 4) arc (180:90:1);
    \draw (4, 4) arc (0:90:1);
    % Arch: upper.
    \draw (1, 5) arc (180:120:1);
    \draw (3, 5) arc (0:60:1);
    \draw (2, {5+sqrt(3)}) arc (0:-60:1);
    \draw (2, {5+sqrt(3)}) arc (-180:-120:1);
\end{tikzpicture}
\end{document}
